% !TEX root = ../main.tex

On the paper by \citet{bos2015novel} Lorem ipsum dolor sit amet, consectetur adipisicing elit, sed do eiusmod
tempor incididunt ut labore et dolore magna aliqua. Ut enim ad minim veniam,
quis nostrud exercitation ullamco laboris nisi ut aliquip ex ea commodo
consequat. Duis aute irure dolor in reprehenderit in voluptate velit esse
cillum dolore eu fugiat nulla pariatur. Excepteur sint occaecat cupidatat non
proident, sunt in culpa qui officia deserunt mollit anim id est laborum.

\begin{figure}[H] % H  means 'FORCE HERE'. other options: h, t, p. 
    \centering
    \includegraphics[height=0.3\linewidth]{images/Profile.png}
    \caption{Sample figure}
    \label{fig:sample}
\end{figure}

\begin{equation} \label{eq:sample}         % equation, gather, align
    E = m C^{2} 
\end{equation}

Where:
\begin{itemize}
    \item[$m$] is the mass of the object in kg.
    \item[$C$] is the speed of light. Value of 3\E{9} m/s.
\end{itemize}

You can align equations on the equal sign, as:
\begin{align*} \label{eq:aligned}         % equation, gather, align
    A       &=   33213021.0
    \\
    B       &=   20    
    \\
    C       &=  \exp{\frac{1}{2}}
    \\
    D       &=  \ln{\frac{1}{2}}
    \\
    E       &=  \log{\frac{1}{2}}
\end{align*}

The \texttt{*} means that it goes unnumbered.

Make sure to use \textbf{custom commands} to simplify repetitive typing of math terms, such as dimensionless parameters:

$$
    A = \frac{1}{2} \Gr \Pr^2
$$





% COMMENTS! Make a line for easier reading
%------------------------------------------------------------------------------
\section{A new section} \label{sec:a_new_section}


This is a list with no spacing between the items and reduced width of line compared to the main body of the report. This can be changed in the \texttt{preamble.tex} file
\begin{itemize}
\item rud exercitation ullamco laboris nisi ut aliquip ex ea commodo
consequat. Duis aute irure dolo
\item One
\item Two
\end{itemize}

%------------------------------------------------------------------------------
\subsection{With subsections} \label{sub:with_subsections}

Chemical compounds written with the \texttt{ce} package, as: CH3OH2 turns into \ce{CH3OH2}

\begin{table}[H]    % H  means 'FORCE HERE' other options: h, t, p. 
    \centering
    %\resizebox{\linewidth}{!}{                % Uncomment if table is too wide
    \caption{Table caption}
    \label{tab:sample_table} \vspace{-0.5em}
    \begin{tabular}{lccr}
        \hline
        a    & b        & c    & d      \\ \hline
        1    & 2000     & 3    &  4     \\
        1    & 2000     & 3    &  4     \\
        1    & 2000     & 3    &  4     \\
        1    & 2000     & 3    &  4     \\
        \hline
    \end{tabular}
    %}                                         % Uncomment if table is too wide
\end{table}


%------------------------------------------------------------------------------
\paragraph{A parapgh label} \label{par:a_parapgh_label}

This is unnumbered, so it's good for simple separations. Reference as Sec.~\ref{sec:a_new_section}. The \texttt{\~} character means \textbf{unbroken space} which will force the words to always be held together across linebreaks


\section{Custom To-do list}
When writing and spotting corrections necessary for later, I like to make notes of them for me to correct them later.

\todos{Fix the alignment of whatever.}

These get listed into the final page of the draft report for easier readability.
\textbf{don't forget to comment out} the \texttt{listoftodos} command
